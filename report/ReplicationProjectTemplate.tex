\documentclass{article}

\usepackage{arxiv}

\usepackage[utf8]{inputenc} % allow utf-8 input
\usepackage[T1]{fontenc}    % use 8-bit T1 fonts
\usepackage{lmodern}        % https://github.com/rstudio/rticles/issues/343
\usepackage{hyperref}       % hyperlinks
\usepackage{url}            % simple URL typesetting
\usepackage{booktabs}       % professional-quality tables
\usepackage{amsfonts}       % blackboard math symbols
\usepackage{nicefrac}       % compact symbols for 1/2, etc.
\usepackage{microtype}      % microtypography
\usepackage{graphicx}

\title{A template for Quantitative Finance Research Replication}

\author{
    First Author
   \\
    Financial Engineering \\
    University of Illinois \\
   \\
  \texttt{\href{mailto:youremail@illinois.edu}{\nolinkurl{youremail@illinois.edu}}} \\
   \And
    Second Author (optional, continue for third, fourth, etc)
   \\
    Financial Engineering \\
    University of Illinois \\
   \\
  \texttt{\href{mailto:youremail@illinois.edu}{\nolinkurl{youremail@illinois.edu}}} \\
  }


% tightlist command for lists without linebreak
\providecommand{\tightlist}{%
  \setlength{\itemsep}{0pt}\setlength{\parskip}{0pt}}


% Pandoc citation processing
\newlength{\cslhangindent}
\setlength{\cslhangindent}{1.5em}
\newlength{\csllabelwidth}
\setlength{\csllabelwidth}{3em}
\newlength{\cslentryspacingunit} % times entry-spacing
\setlength{\cslentryspacingunit}{\parskip}
% for Pandoc 2.8 to 2.10.1
\newenvironment{cslreferences}%
  {}%
  {\par}
% For Pandoc 2.11+
\newenvironment{CSLReferences}[2] % #1 hanging-ident, #2 entry spacing
 {% don't indent paragraphs
  \setlength{\parindent}{0pt}
  % turn on hanging indent if param 1 is 1
  \ifodd #1
  \let\oldpar\par
  \def\par{\hangindent=\cslhangindent\oldpar}
  \fi
  % set entry spacing
  \setlength{\parskip}{#2\cslentryspacingunit}
 }%
 {}
\usepackage{calc}
\newcommand{\CSLBlock}[1]{#1\hfill\break}
\newcommand{\CSLLeftMargin}[1]{\parbox[t]{\csllabelwidth}{#1}}
\newcommand{\CSLRightInline}[1]{\parbox[t]{\linewidth - \csllabelwidth}{#1}\break}
\newcommand{\CSLIndent}[1]{\hspace{\cslhangindent}#1}

\begin{document}
\maketitle


\begin{abstract}
Enter the text of your abstract here. Consider your précis from the
summary and add 1-3 sentences about your conclusions. 6-10 sentences.
\end{abstract}

\keywords{
    trading strategies
   \and
    quantitative analysis
   \and
    machine learning
   \and
    these are optional and can be removed
  }

\hypertarget{introduction}{%
\section{Introduction}\label{introduction}}

This is a template for Research Replication projects following the
process outlined in Peterson (2016). This template uses Rmarkdown (Xie
2017). The sample bibliography file was generated via \emph{Jabref.}
(2021) . The additional comments in this template will be inserted as
comments, and will not be included in the compiled output. Alternately,
Jupyter Notebooks with Executable Books Community (2020) and
(\textbf{jupytext?}) can be used, but I feel that RMarkdown (even with
python code blocks) is easier.

\hypertarget{paper-summary}{%
\section{Paper Summary}\label{paper-summary}}

Start with a single paragraph in précis form. See Peterson (2016) p.~1-2
for details. Complete this section with paragraphs describing each major
point in the paper. The entire summary will be 4-10 paragraphs.

\hypertarget{hypothesis-overview}{%
\section{Hypothesis Overview}\label{hypothesis-overview}}

Formally detail the paper's key hypotheses. See Peterson (2016) p.~2 for
details.

\hypertarget{literature-review}{%
\section{Literature Review}\label{literature-review}}

Write your literature review. See Peterson (2016) p.~2-4 for details.
This section must include paragraphs at least for the 3-5 key references
for the paper to be replicated, similar work, implementation references,
more recent references where available, and any references with attempt
to refute the hypotheses of the replicated work. A full literature
review may contain 20-50 references. Not all will be covered in the same
level of detail. Important references probably warrant an entire
paragraph, but similar work can probably be covered together in 1-2
paragraphs for multiple related works.

\hypertarget{replication}{%
\section{Replication}\label{replication}}

Now we move on to the actual replication. The sections included here are
all necessary, but the may not be sufficient. Add additional sections
and sub-sections as required to describe your work and make your
analytical case.

\hypertarget{data}{%
\subsection{Data}\label{data}}

Describe the approach that the replication is taking to Data. See
Peterson (2016) p.~4-5 for details. Describe both the data used in the
original paper, and the data you are using for replication. For your
replicated data, include detailed descriptions of obtaining, parsing,
and cleaning the data to prepare it for use. Describe data quality
issues.

\hypertarget{replication-of-key-analytical-techniques}{%
\subsection{Replication of Key Analytical
Techniques}\label{replication-of-key-analytical-techniques}}

Model the Key Analytical Techniques from the paper to be replicated. See
Peterson (2016) p.~5-6 for details. This section will vary significantly
based on the paper being replicated. Describe your process as you work,
documenting the steps you are taking, referencing any libraries,
websites, or third party code that you use as part of your replication,
and the decree to which your replication agrees or disagrees with the
source material. Be sure to include summary statistics used in the
original paper, as well as any additional summary statistics that you
feel are relevant for checking the quality of your replication.

\hypertarget{technique-1}{%
\subsubsection{Technique 1}\label{technique-1}}

\hypertarget{technique-2}{%
\subsubsection{Technique 2}\label{technique-2}}

\hypertarget{technique-3}{%
\subsubsection{Technique 3}\label{technique-3}}

\hypertarget{hypothesis-tests}{%
\subsection{Hypothesis Tests}\label{hypothesis-tests}}

After replicating the initial work, it is time to evaluate the
hypotheses of the replicated work. Those hypotheses were identified
above, before you started replication. Describe, in detail, the
statistical tests you perform to refute or validate the hypotheses in
the replicated work. This should go beyond any explicit tests performed
in the original paper.

\hypertarget{extended-analysis}{%
\subsection{Extended Analysis}\label{extended-analysis}}

Extend the analysis with more (recent) data or additional asset classes,
and/or replicate similar or extended techniques and compare them to the
original paper's methods. See Peterson (2016) p.~6-7 for details.

\hypertarget{overfitting}{%
\subsection{Overfitting}\label{overfitting}}

Analyze the likelihood that the original paper is overfit. Include data
considerations, experiment design, model assumptions, parameterization,
and biases, out of sample results, etc. Assess how changes to these
affects results, and produce an opinion on whether and how the original
work is overfit, as well as what might be doable to reduce the degree of
overfitting, and whether the main results would hold if the level of
overfitting were reduced.

\hypertarget{future-work}{%
\section{Future Work}\label{future-work}}

What additional work on this topic should be performed in the future, if
this project were to be picked up again or continued?

\hypertarget{conclusions}{%
\section{Conclusions}\label{conclusions}}

Summarize the project and describe your conclusions. This sections can
range from 1-2 paragraphs to 1-2 pages.

\newpage

\begin{figure}
\centering
\includegraphics{cc_by_88x31.png}
\caption{CC-BY}
\end{figure}

\hypertarget{references}{%
\section*{References}\label{references}}
\addcontentsline{toc}{section}{References}

\hypertarget{refs}{}
\begin{CSLReferences}{1}{0}
\leavevmode\vadjust pre{\hypertarget{ref-jupyterbook}{}}%
Executable Books Community. 2020. \emph{Jupyter Book} (version v0.10).
Zenodo. \url{https://doi.org/10.5281/zenodo.4539666}.

\leavevmode\vadjust pre{\hypertarget{ref-jabref}{}}%
\emph{Jabref.} 2021. \url{https://www.jabref.org}.

\leavevmode\vadjust pre{\hypertarget{ref-PetersonReplication}{}}%
Peterson, Brian G. 2016. {``Research Replication.''}
\url{https://www.researchgate.net/publication/319298241_Research_Replication}.

\leavevmode\vadjust pre{\hypertarget{ref-Peterson2015}{}}%
---------. 2017. {``Developing \& Backtesting Systematic Trading
Strategies.''}
\url{https://www.researchgate.net/publication/319298448_Developing_Backtesting_Systematic_Trading_Strategies}.

\leavevmode\vadjust pre{\hypertarget{ref-Rmarkdown}{}}%
Xie, Yihui. 2017. {``R Markdown --- Dynamic Documents for r.''}
\url{http://rmarkdown.rstudio.com/}.

\end{CSLReferences}

\bibliographystyle{unsrt}
\bibliography{references.bib}


\end{document}
