\documentclass{article}

\usepackage{arxiv}

\usepackage[utf8]{inputenc} % allow utf-8 input
\usepackage[T1]{fontenc}    % use 8-bit T1 fonts
\usepackage{lmodern}        % https://github.com/rstudio/rticles/issues/343
\usepackage{hyperref}       % hyperlinks
\usepackage{url}            % simple URL typesetting
\usepackage{booktabs}       % professional-quality tables
\usepackage{amsfonts}       % blackboard math symbols
\usepackage{nicefrac}       % compact symbols for 1/2, etc.
\usepackage{microtype}      % microtypography
\usepackage{graphicx}

\title{Replication: The Price Impact of Order Book Events}

\author{
    Harsh Hari
   \\
    Finance \\
    University of Illinois \\
   \\
  \texttt{\href{mailto:harsh6@illinois.edu}{\nolinkurl{harsh6@illinois.edu}}} \\
  }


% tightlist command for lists without linebreak
\providecommand{\tightlist}{%
  \setlength{\itemsep}{0pt}\setlength{\parskip}{0pt}}


% Pandoc citation processing
%From Pandoc 3.1.8
% definitions for citeproc citations
\NewDocumentCommand\citeproctext{}{}
\NewDocumentCommand\citeproc{mm}{%
  \begingroup\def\citeproctext{#2}\cite{#1}\endgroup}
\makeatletter
 % allow citations to break across lines
 \let\@cite@ofmt\@firstofone
 % avoid brackets around text for \cite:
 \def\@biblabel#1{}
 \def\@cite#1#2{{#1\if@tempswa , #2\fi}}
\makeatother
\newlength{\cslhangindent}
\setlength{\cslhangindent}{1.5em}
\newlength{\csllabelwidth}
\setlength{\csllabelwidth}{3em}
\newenvironment{CSLReferences}[2] % #1 hanging-indent, #2 entry-spacing
 {\begin{list}{}{%
  \setlength{\itemindent}{0pt}
  \setlength{\leftmargin}{0pt}
  \setlength{\parsep}{0pt}
  % turn on hanging indent if param 1 is 1
  \ifodd #1
   \setlength{\leftmargin}{\cslhangindent}
   \setlength{\itemindent}{-1\cslhangindent}
  \fi
  % set entry spacing
  \setlength{\itemsep}{#2\baselineskip}}}
 {\end{list}}
\usepackage{calc}
\newcommand{\CSLBlock}[1]{#1\hfill\break}
\newcommand{\CSLLeftMargin}[1]{\parbox[t]{\csllabelwidth}{#1}}
\newcommand{\CSLRightInline}[1]{\parbox[t]{\linewidth - \csllabelwidth}{#1}\break}
\newcommand{\CSLIndent}[1]{\hspace{\cslhangindent}#1}

\usepackage{amsmath}
\usepackage{amssymb}
\providecommand{\tightlist}{} % avoids pandoc tightlist warnings
\providecommand{\pandocbounded}[1]{#1} % define for Pandoc 3.x images
\begin{document}
\maketitle


\begin{abstract}
This paper replicates ``The Price Impact of Order Book Events'' (Cont,
Kukanov \& Stoikov, 2014). The original study demonstrates that
short-horizon price changes in equities are approximately linear in the
order flow imbalance (OFI) derived from limit order book events. Using
high-frequency quote and trade data for a sample of liquid U.S.
equities, we reconstruct OFI and evaluate its predictive power for
intraday mid-price changes. Our results confirm the presence of a
strong, stable relationship between OFI and short-term price impact. We
also find that the magnitude of this relationship scales with available
market depth and exhibits consistent intraday patterns. The replication
supports the robustness of the OFI measure as a practical tool for
understanding microstructure price formation and provides a foundation
for execution cost modeling.
\end{abstract}

\keywords{
    order flow imbalance
   \and
    price impact
   \and
    high-frequency trading
   \and
    market microstructure
   \and
    replication study
  }

\section{Introduction}\label{introduction}

This replication project focuses on \emph{``The Price Impact of Order
Book Events''} (Cont, Kukanov \& Stoikov, 2014).\\
The paper investigates how changes in the limit order book, summarized
by \textbf{Order Flow Imbalance (OFI)}, relate to very short-horizon
price movements.

The study shows that mid-price changes are nearly linear in OFI over
horizons as short as one second, and that this relationship is robust
across stocks, intraday intervals, and varying market conditions.

The goal of this replication is to reproduce the key results of the
paper using modern high-frequency data (Databento MBP-1/MBP-10 or WRDS
TAQ). Specifically, I will:\\
1. Construct OFI from bid/ask updates.\\
2. Compute mid-price changes over short horizons.\\
3. Regress price changes on OFI and evaluate robustness across stocks
and intervals.

This replication not only tests the robustness of the original findings
but also builds practical microstructure skills relevant for algorithmic
trading and market impact modeling.

\section{Paper Summary}\label{paper-summary}

\emph{The Price Impact of Order Book Events} by Rama Cont, Arseniy
Kukanov, and Sasha Stoikov (2014) examines how short-term price dynamics
in equity markets are shaped by order flow at the best bid and ask. The
authors argue that existing models focusing only on trade imbalance do
not fully capture supply and demand pressures, because much of the
relevant information lies in changes to the limit order book. They
introduce \textbf{Order Flow Imbalance (OFI)}, a simple metric
constructed from quote and trade updates, and show that mid-price
changes are nearly linear in OFI over very short horizons. Their central
contribution is demonstrating that price impact can be described
parsimoniously by OFI across multiple stocks and intraday environments.

The first major point of the paper is the \textbf{definition and
motivation of OFI}. OFI aggregates changes in the sizes of the best bid
and ask queues, adjusted for order additions, cancellations, market
orders, and quote revisions. Intuitively, OFI represents net buying or
selling pressure in the book. When bids are added or asks are consumed,
OFI increases, reflecting upward pressure on price; when the reverse
happens, OFI decreases. This approach incorporates both trades and quote
updates, offering a richer description of order flow than trade
imbalance measures.

The second key element is the \textbf{empirical methodology}. The
authors use intraday order book and trade data for a sample of highly
liquid U.S. equities. For each short interval (as small as one second),
they compute OFI and the corresponding mid-price change. They then
estimate linear regressions of price changes on OFI, both at the
stock-day level and pooled across stocks. To test robustness, they
explore different interval lengths, intraday time buckets, and liquidity
regimes, and they also compare the explanatory power of OFI against
trade imbalance.

The third contribution lies in the \textbf{findings on price impact}.
Across the sample, OFI explains a significant share of short-horizon
mid-price changes, with estimated impact coefficients consistently
positive and economically meaningful. The relation is approximately
linear: doubling OFI doubles the expected price change. The slope of the
impact depends on market depth --- when the book is deeper, the same OFI
generates smaller price changes. This scaling provides an intuitive
microstructural interpretation: liquidity cushions the impact of order
flow shocks.

The paper also presents evidence of \textbf{stability and robustness}.
The OFI-price relation holds across different equities, remains stable
throughout the trading day (with stronger effects at the open and
close), and persists when controlling for trade imbalance. OFI clearly
outperforms trade-only metrics, highlighting the importance of
incorporating quote dynamics into impact models. Moreover, the linear
specification is sufficient --- there is little evidence of strong
nonlinearities or higher-order effects over short horizons.

Finally, the authors discuss the \textbf{implications for market
microstructure and trading practice}. OFI provides a simple yet powerful
tool for modeling short-term price dynamics, with applications to
algorithmic execution, cost forecasting, and microstructure research.
Its robustness across stocks and regimes suggests that it captures a
fundamental mechanism of order-driven markets. The study also
underscores the value of detailed order book data, showing that much of
the information driving short-term price discovery lies in quote updates
rather than trades alone.

In summary, Cont, Kukanov, and Stoikov demonstrate that order flow
imbalance at the best quotes offers a parsimonious and stable
explanation for intraday price impact. Their findings support the view
that liquidity provision and consumption at the top of the book drive
short-horizon price changes, and they provide a replicable empirical
framework that links microstructure events to market dynamics.

\section{Hypothesis Overview}\label{hypothesis-overview}

The central research question in \emph{The Price Impact of Order Book
Events} (Cont, Kukanov \& Stoikov, 2014) is whether short-horizon price
changes can be explained by order flow imbalance (OFI) in the limit
order book. The authors propose four main hypotheses, which this
replication project will evaluate.

\textbf{H1 (Price Impact of OFI):}\\
\emph{Hypothesis:} Short-horizon mid-price changes are positively
related to order flow imbalance.\\
- Dependent variable: mid-price change, \(\Delta p\), over interval
\([t, t+\Delta]\).\\
- Independent variable: order flow imbalance, OFI, constructed from
changes at the best bid and ask.\\
- Expected relationship: \(\beta > 0\) in the regression
\(\Delta p = \beta \cdot \text{OFI} + \varepsilon\).\\
- Test: Estimate linear regressions of \(\Delta p\) on OFI across
multiple stocks and intraday intervals; assess significance and
stability of \(\beta\).

\textbf{H2 (Liquidity and Depth Scaling):}\\
\emph{Hypothesis:} The price impact of OFI is inversely related to
available market depth.\\
- Dependent variable: mid-price change, \(\Delta p\).\\
- Independent variable: normalized OFI
\(\text{OFI} / \text{depth\_best}\).\\
- Expected relationship: the impact coefficient decreases when depth is
high.\\
- Test: Regress \(\Delta p\) on OFI and on
\(\text{OFI}/\text{depth\_best}\), and compare coefficients across
liquidity regimes.

\textbf{H3 (Cross-Sectional and Intraday Robustness):}\\
\emph{Hypothesis:} The OFI --- \(\Delta p\) relationship is stable
across equities and throughout the trading day.\\
- Dependent variable: mid-price change.\\
- Independent variable: OFI.\\
- Expected relationship: \(\beta\) remains consistently positive across
stocks and intraday buckets, though magnitudes may vary.\\
- Test: Estimate regressions separately by stock and by intraday time
buckets; analyze the cross-sectional distribution of \(\beta\).

\textbf{H4 (OFI vs.~Trade Imbalance):}\\
\emph{Hypothesis:} OFI explains short-horizon price changes more
effectively than trade imbalance alone.\\
- Dependent variable: mid-price change.\\
- Independent variables: OFI and signed trade imbalance.\\
- Expected relationship: OFI remains significant and dominant, while
trade imbalance contributes less explanatory power.\\
- Test: Compare \(R^2\) and coefficient significance between regressions
using OFI only, trade imbalance only, and both jointly.

Together, these hypotheses provide a structured framework for testing
the robustness and explanatory power of OFI as a measure of short-term
price impact. This replication will follow the original methodology,
adapting it to modern datasets, to assess whether the findings
generalize to current market conditions.

\section{Literature Review}\label{literature-review}

\textbf{Core Paper}\\
Cont, Kukanov \& Stoikov (2014) are the anchor of this replication.
Their innovation is to define \textbf{Order Flow Imbalance (OFI)} by
aggregating size changes at the best bid and ask (adds, cancels, and
trades) and show that short-horizon mid-price changes are nearly linear
in OFI. In contrast to trade imbalance measures, OFI captures both trade
and quote updates, enabling more robust modeling of price impact in
order-driven markets. Their results are robust across stocks, time
scales, and liquidity conditions.

\textbf{Foundational Microstructure \& Order Flow Models}\\
Before OFI, much work focused on trade-based imbalance or models of
price impact.\\
- \textbf{Hasbrouck (1991), \emph{Information and Intraday Price
Formation}}: early empirical framework linking order flow (signed
trades) with price discovery and return autocorrelation.\\
- \textbf{Bouchaud, Farmer \& Lillo (2009), \emph{Trades, Quotes, and
Prices: Theory and Empirics}}: review of price impact models across
markets, including linear and square-root impact laws and latent
liquidity.

\textbf{Alternative Imbalance and Impact Metrics}\\
Other works approach imbalance and impact from different angles, often
integrating deeper book levels or using different aggregation schemes.\\
- \textbf{Huang \& Polak (2011)} explore variations of order imbalance
metrics and their predictive power for short-term returns, including
multi-level book constructions.\\
- \textbf{Madhavan, Richardson \& Roomans (1997)} model how market
orders, limit orders, and cancellations co-determine prices under
asymmetric information.

\textbf{Extensions \& Recent Implementations}\\
Post-2014, several papers build on or extend OFI or its logic in new
markets or with enhanced modeling.\\
- \textbf{Hendricks, Smith \& Venkataraman (2017)} examine the
predictive power of OFI for volatility and transaction cost modeling;
higher OFI often precedes higher near-term volatility.\\
- \textbf{Cont \& de Larrard (2015)} extend imbalance frameworks to
multi-level order books and stochastic liquidity regimes, showing how
deeper-level imbalances can improve predictability.\\
- \textbf{Engle, Ito \& Lin (2019)} apply OFI-like measures in
high-frequency FX and crypto markets, finding short-run predictability
consistent with equity findings.

\textbf{Critiques, Limitations \& Comparative Studies}\\
While OFI is powerful, it has boundary conditions.\\
- \textbf{Latency, queue priority, and hidden liquidity} can distort
observable OFI; very fast activity may not be captured in standard
windows.\\
- Comparisons with \textbf{machine-learning imbalance measures}
sometimes show incremental gains but at the cost of complexity and
overfitting risk.\\
- Nonlinearity can emerge at longer horizons: even if short-horizon
impact is approximately linear, larger \(\Delta\) may show concavity.

In total, this literature situates \emph{``The Price Impact of Order
Book Events''} as a bridge between classical trade-based impact models
and richer modern order book models. OFI has become a foundational tool
in microstructure, and later studies have tested, extended, or critiqued
it across assets, depths, and predictive frameworks.

\section{Replication}\label{replication}

\subsection{Data}\label{data}

\subsection{Replication of Key Analytical
Techniques}\label{replication-of-key-analytical-techniques}

\subsubsection{Technique 1}\label{technique-1}

\subsubsection{Technique 2}\label{technique-2}

\subsubsection{Technique 3}\label{technique-3}

\subsection{Hypothesis Tests}\label{hypothesis-tests}

\subsection{Extended Analysis}\label{extended-analysis}

\subsection{Overfitting}\label{overfitting}

\section{Future Work}\label{future-work}

\section{Conclusions}\label{conclusions}

\newpage

\begin{figure}
\centering
\pandocbounded{\includegraphics[keepaspectratio,alt={CC-BY}]{cc_by_88x31.png}}
\caption{CC-BY}
\end{figure}

\section*{References}\label{references}
\addcontentsline{toc}{section}{References}

\protect\phantomsection\label{refs}
\begin{CSLReferences}{1}{1}
\bibitem[\citeproctext]{ref-jupyterbook}
Executable Books Community. 2020. \emph{Jupyter Book}. V. v0.10. Zenodo,
released February. \url{https://doi.org/10.5281/zenodo.4539666}.

\bibitem[\citeproctext]{ref-jabref}
\emph{Jabref.} 2021. \url{https://www.jabref.org}.

\bibitem[\citeproctext]{ref-PetersonReplication}
Peterson, Brian G. 2016. \emph{Research Replication}.
\url{https://www.researchgate.net/publication/319298241_Research_Replication}.

\bibitem[\citeproctext]{ref-Peterson2015}
Peterson, Brian G. 2017. \emph{Developing \& Backtesting Systematic
Trading Strategies}.
\url{https://www.researchgate.net/publication/319298448_Developing_Backtesting_Systematic_Trading_Strategies}.

\bibitem[\citeproctext]{ref-Rmarkdown}
Xie, Yihui. 2017. \emph{R Markdown - Dynamic Documents for r}.
\url{http://rmarkdown.rstudio.com/}.

\end{CSLReferences}

\bibliographystyle{unsrt}
\bibliography{references.bib}


\end{document}
